\begin{table*}
  \setlength{\tabcolsep}{4pt}
  \scriptsize
\centering
\begin{tabular}{lcccccc}
\toprule
                     & $A \rightarrow D$   & $A \rightarrow W$   & $D \rightarrow A$   & $D \rightarrow W$   & $W \rightarrow A$   & $W \rightarrow D$   \\
\midrule
GFK(PLS,PCA)~\cite{gong-cvpr12} & - & 15.0$\pm$0.4 & - & 44.6$\pm$0.3 & - & 49.7$\pm$0.5\\
SA~\cite{fernando-iccv13} & - & 15.3 & - & 50.1& - & 56.9\\
DA-NBNN~\cite{da-nbnn} & - & 23.3$\pm$2.7 & - & 67.2$\pm$1.9 & - & 67.4$\pm$3.0\\
DLID~\cite{ref:dlid} & - & 26.13 & - & 68.93 & - & 84.94\\                     
\midrule
 Ours (source only)   & $\bm{50.17 \pm 0.6}$     & $45.61 \pm 0.5$     & $45.32 \pm 0.3$     & $\bm{86.52 \pm 0.3}$     & $\bm{44.24 \pm 0.3}$     & $\bm{87.96 \pm 0.4}$     \\
Ours (early fusion)   & $50.07 \pm 0.4$     & $\bm{46.63 \pm 0.7}$     & $\bm{45.54 \pm 0.4}$     & $86.18 \pm 0.3$     & $42.97 \pm 0.4$     & $86.73 \pm 0.5$     \\
\bottomrule
\end{tabular}

\caption{Multi-class accuracy evaluation on the standard unsupervised adaptation setting with the \emph{Office} dataset. We evaluate on all 31 categories using the standard experimental protocol from ~\cite{gong-cvpr12}. Here, we compare against four state of the art domain adaptation methods.(All methods reported on only 3/6 of the domain shifts).}
\label{table:full-unsuper}
\end{table*}

