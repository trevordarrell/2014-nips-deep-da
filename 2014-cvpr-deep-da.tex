\documentclass{article} % For LaTeX2e
\usepackage{nips14submit_e,times}
\usepackage{hyperref}
\usepackage{url}
%\documentstyle[nips14submit_09,times,art10]{article} % For LaTeX 2.09

\usepackage{bm}
\usepackage{latexsym}
\usepackage{amsmath}
\usepackage{amssymb}
\usepackage{booktabs}
\usepackage{epsfig}
\usepackage{graphicx}
%\usepackage{subfigure}
\usepackage{xspace}
\usepackage{wrapfig}
\usepackage{caption}
\usepackage{subcaption}
%\documentstyle[nips13submit_09,times,art10]{article} % For LaTeX 2.09


%\title{Do Supervised Deep Image Models Trained with a Large Dataset Remove Dataset Bias?}
\title{Adapting Supervised Deep Convolutional Models to New Domains}


\author{
%Judy Hoffman, 
Eric Tzeng, Judy Hoffman, Jeff Donahue \\
UC Berkeley, EECS \& ICSI\\
\footnotesize{\texttt{\{etzeng,jhoffman,jdonahue\}@eecs.berkeley.edu} }\\
\And
%Yangqing Jia\thanks{This work was completed while Yangqing Jia was a graduate student at UC Berkeley} \\
%Google Research \\
%\texttt{jiayq@google.com} \\
%\AND
Kate Saenko \\
UMass Lowell, CS  \\
\footnotesize{\texttt{saenko@cs.uml.edu}} \\
\And
Trevor Darrell \\
UC Berkeley, EECS\\
\footnotesize{\texttt{trevor@eecs.berkeley.edu}} \\
}

% The \author macro works with any number of authors. There are two commands
% used to separate the names and addresses of multiple authors: \And and \AND.
%
% Using \And between authors leaves it to \LaTeX{} to determine where to break
% the lines. Using \AND forces a linebreak at that point. So, if \LaTeX{}
% puts 3 of 4 authors names on the first line, and the last on the second
% line, try using \AND instead of \And before the third author name.

\newcommand{\fix}{\marginpar{FIX}}
\newcommand{\new}{\marginpar{NEW}}

\newcommand{\jh}[1]{\textcolor{red}{Judy: #1}}
\newcommand{\ks}[1]{\textcolor{magenta}{Kate: #1}}
\newcommand{\jd}[1]{\textcolor{green}{Jeff: #1}}
\newcommand{\et}[1]{\textcolor{blue}{Eric: #1}}

\newcommand{\daume}{Daum\'e~III\xspace}

\nipsfinalcopy % Uncomment for camera-ready version

\begin{document}


\maketitle

\begin{abstract}
%Dataset bias has been a significant barrier when solving real world computer vision tasks.
%Though deep convolutional networks have proven to be a competitive approach for image classification, a question remains: have these models solved the dataset bias problem?
Recent reports suggest that  a generic supervised deep CNN model trained on a large-scale dataset reduces, but does not remove, dataset bias on a standard benchmark. Fine-tuning deep models in a new domain can require a significant amount of data, which for many applications is simply not available. Here
we propose novel methods for performing adaptation of deep models with limited training data.
These methods make use of a domain similarity metric to adaptively select a deep representation that minimizes the shift between visual domains, thereby enabling adaptation with very little labeled domain specific data.
Our proposed adaptation methods offers empirical performance which exceeds previously published results on a standard benchmark visual domain adaptation task.

\end{abstract}

\section{Introduction}
\input introduction-cvpr

%\vspace{-.3cm}
%\section{Background: Deep Domain Adaptation Approaches}
%\vspace{-.2cm}
%\input decaf

\vspace{-.3cm}
\section{Representation Selection using A-distance}
\label{sec:adapt-algs}
\vspace{-.2cm}
\input method

\section{Evaluation}
Our evaluation follows the protocols in earlier reports \cite{deeplearning-arxiv-2013,hoffman-iclr14}, which we repeat in part below.
\input eval

\section{Conclusion}
\input conclusion


% References
\small{
\bibliographystyle{plain}
\bibliography{main}
}


\end{document}
