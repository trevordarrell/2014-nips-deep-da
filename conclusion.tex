% conclusion
In this paper, we presented the first evaluation of domain adaptation from a
large-scale source dataset with deep features. We demonstrated that, although
using ImageNet as a source domain generalizes better than other smaller source
domains, there is still a domain shift when adapting to other visual domains.

We also proposed a simple yet novel deep domain adaptation framework, as well as
three particular methods DFE, DLF, and DSA, inspired by classic domain
adaptation methods~\cite{daume, sa}. Our methods incorporate A-distance to
non-parametrically select a layer that will provide strong adaptation
performance. We showed that our methods outperform the standard method of
adapting a convolutional network via fine tuning in the setting where a limited
number of labeled target examples are available. We also demonstrated that our
methods achieve state-of-the-art performance on the standard visual domain
adaptation benchmark.

\et{Discuss future work. Novel categories, probably. What else?}
