% conclusion
In this paper, we presented the first evaluation of domain adaptation from a
large-scale source dataset with deep features. We demonstrated that, although
using ImageNet as a source domain generalizes better than other smaller source
domains, there is still a domain shift when adapting to other visual domains.
Regardless, except in cases where there exists a small source domain that is
very similar to the target domain, we find it is more effective to adapt with
a large-scale dataset such as ImageNet.

We also proposed a simple yet novel deep domain adaptation framework, as well as
three particular methods DFE, DLF, and DSA, inspired by classic domain
adaptation methods~\cite{daume, sa}. Our methods incorporate A-distance to
non-parametrically select a layer that will provide strong adaptation
performance. Our methods outperform the standard method of adapting a
convolutional network in the setting where too few labeled target examples are
available for conventional fine-tuning. We also demonstrated that our methods
achieve state-of-the-art performance on the standard visual domain adaptation
benchmark, and that a single target example can boost performance 14\% when
adapting a deep CNN from ImageNet to a small task dataset.

In the future we would like to use our framework in combination with category invariant adaptation methods. This would enable us to evaluate performance in the realistic scenario of having labeled target examples for a subset of the categories we want to classify. 
