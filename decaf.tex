% Decaf description
\label{sec:decaf}
For our task we consider adapting between a large source domain and a target domain with few or or no labeled examples. 
A typical approach to domain adaptation or transfer learning with deep architectures is to take the representation learned via back-propagation on a large dataset, and then transfer the representation to a smaller dataset by fine-tuning, i.e. backpropagation at a lower learning rate \cite{rcnn,zeiler-arxiv-2013}.
However, fine-tuning requires an ample amount of labeled target data and so should not be expected to work well when we consider the very sparse label condition, such as the \textit{one-shot learning} scenario we evaluate below, where we have just one labeled example per category in the target domain.

In fact, in our experiments under this setting, fine-tuning, with the standard approach, actually reduces performance.
Specifically, on the ImageNet$\rightarrow$Webcam task reported in Section~\ref{sec:eval}, using the final output layer as a predictor in the target domain received 66\% accuracy, while using the final output layer after fine tuning produced a degraded accuracy of 61\%. In our experiments, the only clear way to avoid overfitting and reducing overall performance was to set the learning rates so low the model essentially did not change. This is unsurprising, as fine tuning normally requires many additional labeled examples to be effective. \et{These numbers are wrong -- not sure what to do.}

A separate method that was recently proposed for deep adaptation is called Deep Learning for domain adaptation by Interpolating between Domains (DLID)~\cite{ref:dlid}. This method learns multiple unsupervised deep models directly on the source, target, and combined datasets and uses a representation which is the concatenation of the outputs of each model as its adaptation approach. 
While this was shown to be an interesting approach, it is limited by its use of only convolutional deep layers and its focus on unsupervised learning for the majority of its training.

In general, unsupervised deep convolutional models have been unable to achieve the performance of supervised deep CNNs. However, training a supervised deep model requires sufficient labeled data. Our insight is that the extensive labeled data available in the source domain can be exploited using a supervised model without requiring a significant amount of labeled target data.

Therefore, we propose learning a supervised deep CNN using a large amount of source data. Then, we considering fixing the network and training a new final adaptation layer that takes as input the activations from both the source and target data. By using this hybrid approach we utilize the strong representation available from the supervised deep model trained on a large source dataset while requiring only enough target labeled data to train a shallow model with far fewer parameters. 


%For extracting features from the deep source model, we follow the setup of Donahue et al.~\cite{deeplearning-arxiv-2013}, which extracts a visual feature \textit{DeCAF} from the ImageNet-trained architecture of~\cite{supervision}.

% Following the successful experiments of~\cite{deeplearning-arxiv-2013}, our experiments in Section~\ref{sec:eval} take features computed from the fully-connected layers of the network: DeCAF$_6$ and DeCAF$_7$.
% We then go beyond the ``mid-level'' layers to explore DeCAF$_8$, the 1000-dimensional layer whose activations are fed into a soft-max unit to compute the label probabilities, as a basis for adaptation.


%Convolutional neural networks (CNNs) are a layered representation for learning features from and predicting statistics of visual data.
%CNN architectures had early practical success in handwritten digit recognition via supervised back-propagation~\cite{lecun89}.
%Powerful GPUs and GPU implementations accompanied by large datasets have enabled high-capacity CNNs to emerge as the current state-of-the-art visual classification method, with~\cite{supervision} demonstrating that an eight-layer architecture with over 60 million parameters achieves a top-5 error of 18.2\% on the ILSVRC-2012~\cite{ilsvrc2012} validation set, sparking great interest in this particular network.
%The network is trained by fully supervised back-propagation (as in~\cite{lecun89}) from raw RGB image pixels (of a fixed size of 224 $\times$ 224) to the desired output labels.
%The first five layers of the network are convolutional; i.e., the activation of any particular neuron is determined entirely by the pixel values within a particular window of the input image, and the filter weights are shared across every input pixel neighborhood to which the filter is applied.
%Each of the final three layers, on the other hand, are fully-connected to each neuron from the previous layer, making features taken from these layers ``global'' in the sense that they (potentially) depend on the entire input image.
