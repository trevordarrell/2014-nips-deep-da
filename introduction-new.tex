
Supervised deep convolutional neural networks (CNNs) trained on large-scale
classification tasks have been shown to learn impressive mid-level structures
and obtain high levels of performance on contemporary classification
challenges \cite{ilsvrc2012,zeiler-arxiv-2013}. These models generally assume
extensive training using labeled data, and testing is limited to data from the
same domain. In practice, however, the images we would like to classify are
often produced under different imaging conditions or drawn from a different
distribution, leading to a domain shift. Scaling such models to new domains
remains an open challenge.

Deep CNNs require large amounts of training data to learn
good mid-level convolutional models and final fully-connected classifier
stages. While the continuing expansion of web-based datasets like
ImageNet \cite{ilsvrc2012} promises to produce labeled data for almost any desired
category, such large-scale supervised datasets
may not include images of the category across all
domains of practical interest. Earlier deep learning efforts addressed this
challenge by learning layers in an unsupervised fashion using unlabeled data to
discover salient mid-level structures \cite{coates-nips12, dean-nips12}. While such approaches are appealing, they
have heretofore been unable to match the level of performance of supervised
models, and unsupervised training of networks with the same level of depth
as \cite{supervision} remains a challenge.

 
Unfortunately, supervised image datasets are inherently biased \cite{efros-cvpr11}. 
Theoretical \cite{ben2007analysis, blitzer2007learning} and practical results \cite{saenko-eccv10,efros-cvpr11} have shown that supervised methods' test error increases in proportion to the difference between the test and training input distribution. 
Many visual domain adaptation methods have been put forth to compensate for dataset bias \cite{daume,yang-icdm07,aytar-iccv11,saenko-eccv10,kulis-cvpr11,Khosla-eccv12,gopalan-iccv11,gong-cvpr12,hoffman-eccv12,hoffman-iclr13}, but are limited to shallow models. For deep models, fine-tuning is the most prevalent approach for supervised adaptation to novel tasks (see \cite{rcnn} for a recent example). 

In this paper, we ask,
%\textit{Is fine-tuning the answer to domain adaptation with limited labeled target data?}
\textit{Is fine-tuning the answer to the problem of domain shift with limited labeled target data?}
In fact, fine-tuning a deep network's parameters on a small amount of labeled target data turns out to be unsurprisingly
problematic. We instead propose a simple yet intuitive adaptation framework that integrates existing shallow domain adaptation techniques with the deep architecture.
First, we use the A-distance metric~\cite{adist} to select the layer from the
pre-trained CNN that minimizes the shift between the source and target domains.  We then learn a final domain-adapted output layer using the activations
from the selected network layer.  
We provide a comprehensive evaluation on the popular Office benchmark for classification across visually distinct domains, which contains 31 image categories and 3 domains \cite{saenko-eccv10}. 

We also ask, \textit{Is adaptation even necessary for deep CNNs trained on truly large source datasets?}
Recently, \cite{deeplearning-arxiv-2013} showed that using the deep mid-level features learned on ImageNet, instead of the more conventional bag-of-words features, effectively removed the bias in some of the domain adaptation settings in the Office dataset \cite{saenko-eccv10}.
However, while \cite{deeplearning-arxiv-2013} transferred the representation from a large scale domain, they limited the labeled category examples to small source domains found in Office, 
which are three orders of magnitude smaller than ImageNet. 
%They also hand-select the depth of representation layers, whereas we provide an automatic way to select the depth of the network that is most useful for adaptation.
To answer this question, we provide the first evaluation of benchmark visual domain adaptation tasks with deep learned representations in its most natural setting, with truly large-scale domain used as the source of category labels.
Recent work by Rodner et al.~\cite{rodner-arxiv13} attempted to adapt from ImageNet to the SUN dataset, but did not take advantage of deep convolutional features. 

We use the 1.2 million
labeled images available in the 2012 ImageNet 1000-way classification
dataset~\cite{ilsvrc2012} to train the representation in \cite{supervision}, and about 1200 labeled images per category as the source of category labels.
This constitutes an order of magnitude increase in category-labeled source data compared to the approximately 100 labeled images per category of the largest domain in Office.  We find that in most cases it is better to adapt from large-scale source domains, but that dataset bias nonetheless remains a major issue.
The good news is, when adapting
from ImageNet to Office, it turns out to be possible to achieve target domain
performance on par with source domain performance using only a single labeled
example per target category (one-shot adaptation).  Our method also exceeds all previously published
results on the standard benchmark Office adaptation task.

We examine both the setting where there are a few labeled examples from the target domain (\emph{supervised adaptation}) and the setting where there are no labeled target examples (\emph{unsupervised adaptation}). We also describe practical solutions for choosing between the various adaptation methods based on experimental constraints such as limited computation time. 
%quickly adding a new category at test time without retraining the large dataset, etc.
